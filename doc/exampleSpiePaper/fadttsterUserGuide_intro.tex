\documentclass[fadttsterUserGuide_master]{subfiles}

\begin{document}
	\part{What is FADTTSter?}
	The analysis of brain pathologies and development heavily rely on diffusion tensor imaging (DTI) to assess white matter integrity or maturation. The UNC-Utah NA-MIC DTI framework is end-to-end tools set for atlas fiber tract based DTI analysis \citep{unc-utah_namic}.\\
	
	add figure with framework steps\\
	
	During the statistical analysis of diffusion properties, Functional Analysis of Diffusion Tensor Tract Statistic (FADTTS) is computed using fiber bunble profiles obtained from DTIAtlasFiberAnalyzer as inputs.
	FADTTS is Matalb\textsuperscript{\textregistered} (MathWorks Inc, MA, USA) based. The script must be manually updated to fit every study. Therefore, previous knowledge of coding is necessary to operate it.
	
	\newpage	
	FADTTSter was first created to overcome this issue and make the statistical analysis accessible to any non-technical researcher. Now, FADTTSter is even more developed and features very useful options such as subjects management, profile croping, data plotting, ...

	FADTTSter can be divided in two main parts, each one working independently.
	During Matalb\textsuperscript{\textregistered} script generation, the .m script is automatically generated, as well as its inputs, based on the information provided by the (diffusion profiles, subjects, qc threshold, nbr of permutations, p-value threshold, ...).
	Statistical data plotting enables the visualization of the data obtained after running the Matalb\textsuperscript{\textregistered} script and the customization of the plot (title, colors, legend, ...).\\
	
	add figure with FADTTSter box\\
	
	Not only is FADTTSter practical but it enables any investigator to perform DTI analysis efficiently.
	This work has been motivated by research applications in which many fibers, particularly long ones, are to be analyzed.
\end{document}
